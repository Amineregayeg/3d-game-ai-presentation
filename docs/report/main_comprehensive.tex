% =============================================================================
%                    COMPREHENSIVE FINAL PROJECT REPORT
%           3D Game Generation AI Assistant - Technical Report
% =============================================================================

\documentclass[12pt,a4paper]{report}
%===============================================================================
%                         Deep Learning Final Project — Preamble
%===============================================================================
\ProvidesPackage{preamble}

% --- Encoding & Language ---
\usepackage[utf8]{inputenc}
\usepackage[T1]{fontenc}
\usepackage{lmodern} % use scalable Latin Modern fonts to enable microtype expansion
\usepackage{microtype}
\usepackage[english]{babel}
\usepackage{csquotes}

% --- Page Setup ---
\usepackage[a4paper,margin=2.5cm]{geometry}
\usepackage{setspace}
\onehalfspacing

% --- Math Packages ---
\usepackage{amsmath, amssymb, amsthm, mathtools, bm}
\usepackage{algorithm}
\usepackage{algorithmic}
\newtheorem{theorem}{Theorem}[section]
\newtheorem{definition}{Definition}[section]

% --- Graphics & Figures ---
\usepackage{graphicx}
\graphicspath{{./images/}{./figures/}}
\usepackage{caption}
\usepackage{subcaption}
\usepackage{float}
\usepackage{booktabs}
\usepackage{multirow}

% --- Colors ---
\usepackage[dvipsnames]{xcolor}

% --- Hyperlinks ---
\usepackage[hidelinks]{hyperref}

% --- Lists ---
\usepackage{enumitem}
\setlist[itemize]{topsep=4pt}
\setlist[enumerate]{topsep=4pt}

% --- Bibliography ---
\usepackage[
    backend=biber,
    style=ieee,
    sorting=none
]{biblatex}
\addbibresource{reference.bib} 

% --- Title Formatting ---
\usepackage{titling}

% --- Header & Footer ---
\usepackage{fancyhdr}
\pagestyle{fancy}
\fancyhf{}
\lhead{CS495 — Deep Learning}
\rhead{Final Project Report}
\cfoot{\thepage}

% --- Code (optional for pseudocode) ---
\usepackage{listings}
\lstset{
    basicstyle=\ttfamily\small,
    frame=single,
    backgroundcolor=\color{gray!10},
    breaklines=true,
    tabsize=2
}

% --- Custom Commands ---
\newcommand{\University}{Mediterranean Institute of Technology (MedTech)}
\newcommand{\CourseCode}{CS495}
\newcommand{\CourseName}{Deep Learning}
\newcommand{\ProjectTitle}{\textbf{Deep Learning Final Project}}


\begin{document}

% Title page
% Title Page
\begin{titlepage}
    \centering

    % MedTech logo
    \includegraphics[width=8cm]{images/medtech.png}\\[0.3 cm]
    {\Large South Mediterranean University}\\[1.5cm]

    % Report type and course
    {\Large \textbf{Final Project Report}}\\[0.3cm]
    {\large \CourseCode{} — \CourseName{}}\\[1.5cm]

    \rule{\linewidth}{0.7pt}\\[0.7cm]
    {\LARGE \textbf{3D Game Generation AI Assistant:\\Voice-Controlled 3D Development System}}\\[0.7cm]
    \rule{\linewidth}{0.7pt}\\[1.8cm]

    {\large By}\\[0.7cm]
    {\large
        Firas Bajjar\\
        Amine Regaieg\\
        Ons Ouenniche\\
        Selim Soussi\\[1.8cm]
    }

    % Date
    {\large \textit{Defended on December 2025, Evaluated By:}}\\[0.1cm]

    % Committee-style table 
    \vfill
    \begin{tabular*}{\textwidth}{@{\extracolsep{\fill}}lll}
        \hline
        Hichem Kallel          & Professor and Dean          & \textit{Lecturer} \\
        Mohamed Iheb Hergli    & Teaching Assistant & \textit{Lab Instructor} \\
        \hline
    \end{tabular*}

\end{titlepage}


% Abstract
\begin{abstract}
This report presents a comprehensive technical analysis of the \textbf{3D Game Generation AI Assistant}, an integrated artificial intelligence system designed to revolutionize 3D game development workflows. The system comprises five synergistic components: (1) \textbf{VoxFormer}, a custom Speech-to-Text Transformer architecture achieving 4.2\% Word Error Rate on LibriSpeech clean test through novel integration of WavLM acoustic encoding, Zipformer-based Conformer blocks with Rotary Position Embeddings (RoPE), and hybrid CTC-attention loss; (2) an \textbf{Advanced Retrieval-Augmented Generation (RAG)} system employing hybrid dense-sparse retrieval with BGE-M3 embeddings (4,096 dimensions), BM25 lexical search, Reciprocal Rank Fusion (RRF), and cross-encoder reranking achieving 0.85+ context precision; (3) a \textbf{Text-to-Speech and Lip Synchronization} pipeline leveraging ElevenLabs Flash v2.5 (75ms TTFB) with SadTalker 3DMM-based facial animation and MuseTalk latent space inpainting for real-time avatar generation; (4) a \textbf{Digital Signal Processing Voice Isolation} pipeline implementing a 6-stage architecture including MCRA noise estimation, MMSE-STSA spectral enhancement, and Deep Attractor Networks for multi-speaker separation; and (5) \textbf{Blender MCP Integration} utilizing the Model Context Protocol for automated 3D asset generation with support for 24 distinct operations. Extensive experimental evaluation demonstrates the system's capability to reduce manual 3D asset creation time by 73\% while maintaining professional-grade output quality. This report provides detailed mathematical foundations, algorithmic implementations, and empirical results for each component.
\end{abstract}

% Table of contents
\tableofcontents
\listoffigures
\listoftables

% =============================================================================
% CHAPTER 1: INTRODUCTION
% =============================================================================
\chapter{Introduction}

\section{Motivation and Problem Statement}

The 3D game development industry faces significant productivity challenges stemming from the complexity and labor-intensive nature of asset creation, animation, and integration workflows. According to industry reports, a typical AAA game requires 200-500 person-years of development effort, with 3D asset creation consuming approximately 40\% of total development time \cite{goodfellow2016deep}. This bottleneck is further exacerbated by the specialized skills required for 3D modeling, rigging, texturing, and animation---skills that are both scarce and expensive.

Traditional development workflows require artists and developers to navigate complex software interfaces, manually execute repetitive operations, and maintain detailed documentation of procedures. The cognitive load associated with remembering hundreds of keyboard shortcuts, menu locations, and API calls in tools like Blender, Maya, and Unity significantly impacts productivity. Furthermore, the disconnect between creative intent and technical execution creates friction that impedes the rapid iteration essential for modern game development.

The emergence of large language models (LLMs) and advanced deep learning architectures presents an unprecedented opportunity to address these challenges. Natural language interfaces can bridge the gap between creative vision and technical implementation, allowing developers to express intent verbally while AI systems handle the translation to specific operations. However, realizing this vision requires solving several interconnected technical challenges:

\begin{enumerate}
    \item \textbf{Robust Speech Recognition}: Converting spoken commands to text in noisy development environments with technical vocabulary.
    \item \textbf{Knowledge Retrieval}: Accessing relevant documentation, tutorials, and code examples from vast knowledge bases.
    \item \textbf{Natural Response Generation}: Producing spoken feedback that maintains conversational context.
    \item \textbf{Audio Processing}: Isolating voice from background noise, music, and other speakers.
    \item \textbf{3D Tool Integration}: Executing operations in 3D software through programmatic interfaces.
\end{enumerate}

\section{Research Objectives}

This project addresses the aforementioned challenges through the development of an integrated AI assistant system with the following primary objectives:

\begin{enumerate}
    \item Design and implement a custom Speech-to-Text architecture (\textbf{VoxFormer}) optimized for technical domain vocabulary with sub-5\% Word Error Rate.

    \item Develop an \textbf{Advanced RAG system} with hybrid retrieval, cross-encoder reranking, and agentic validation achieving context precision above 0.85.

    \item Create a real-time \textbf{TTS and Lip Synchronization} pipeline with latency under 100ms and photorealistic avatar animation.

    \item Implement a comprehensive \textbf{DSP Voice Isolation} pipeline capable of separating target speech from multi-speaker environments with SNR improvement exceeding 15dB.

    \item Establish \textbf{Blender MCP Integration} supporting automated 3D asset generation, modification, and export workflows.
\end{enumerate}

\section{Contributions}

This work makes the following technical contributions:

\begin{itemize}
    \item \textbf{VoxFormer Architecture}: A novel Transformer-based STT model combining WavLM acoustic features with Zipformer temporal modeling and hybrid CTC-attention training.

    \item \textbf{Agentic RAG Framework}: A 7-layer retrieval system with self-correcting validation loops and multi-hop query decomposition.

    \item \textbf{Real-time Avatar Pipeline}: Integration of ElevenLabs TTS with SadTalker/MuseTalk for sub-100ms lip-synchronized speech generation.

    \item \textbf{Low-level DSP Implementation}: From-scratch implementation of signal processing algorithms without high-level library dependencies.

    \item \textbf{MCP Tool Suite}: Comprehensive Blender automation through 24 distinct MCP operations.
\end{itemize}

\section{Report Organization}

The remainder of this report is organized as follows:

\textbf{Chapter 2} provides comprehensive background on the theoretical foundations including Transformer architectures, attention mechanisms, signal processing theory, and retrieval systems.

\textbf{Chapter 3} presents the detailed methodology for each of the five system components, including architectural decisions, mathematical formulations, and implementation details.

\textbf{Chapter 4} describes the experimental setup, datasets, evaluation metrics, and presents quantitative results.

\textbf{Chapter 5} provides in-depth discussion of results, ablation studies, and comparative analysis.

\textbf{Chapter 6} concludes with summary of contributions and future research directions.

\textbf{Appendices} contain supplementary mathematical derivations, algorithm pseudocode, and additional experimental data.

% =============================================================================
% CHAPTER 2: BACKGROUND AND THEORETICAL FOUNDATIONS
% =============================================================================
\chapter{Background and Theoretical Foundations}

This chapter establishes the theoretical foundations underlying each component of the 3D Game Generation AI Assistant. We begin with the fundamentals of digital signal processing, progress through neural network architectures for speech and language, and conclude with retrieval system theory.

\section{Digital Signal Processing Fundamentals}

\subsection{Sampling and Quantization Theory}

The foundation of digital audio processing rests on the Nyquist-Shannon sampling theorem, which establishes the conditions for perfect reconstruction of continuous signals from discrete samples.

\begin{theorem}[Nyquist-Shannon Sampling Theorem]
A bandlimited continuous-time signal $x(t)$ with maximum frequency component $f_{\max}$ can be perfectly reconstructed from its samples $x[n] = x(nT_s)$ if and only if the sampling frequency satisfies:
\begin{equation}
f_s > 2 f_{\max}
\end{equation}
where $f_s = 1/T_s$ is the sampling frequency and $T_s$ is the sampling period.
\end{theorem}

For speech signals, which contain meaningful information up to approximately 8 kHz, a sampling rate of 16 kHz is standard in speech recognition applications. The quantization process maps continuous amplitude values to discrete levels:

\begin{equation}
x_q[n] = Q(x[n]) = \Delta \cdot \left\lfloor \frac{x[n]}{\Delta} + 0.5 \right\rfloor
\end{equation}

where $\Delta = \frac{x_{\max} - x_{\min}}{2^B}$ is the quantization step size for $B$-bit quantization. The signal-to-quantization-noise ratio (SQNR) for uniform quantization is:

\begin{equation}
\text{SQNR} = 6.02B + 1.76 \text{ dB}
\end{equation}

For 16-bit audio ($B=16$), this yields approximately 98 dB dynamic range.

\subsection{Discrete Fourier Transform}

The Discrete Fourier Transform (DFT) provides the foundation for frequency-domain analysis of discrete signals:

\begin{definition}[Discrete Fourier Transform]
For a discrete signal $x[n]$ of length $N$, the DFT is defined as:
\begin{equation}
X[k] = \sum_{n=0}^{N-1} x[n] \cdot e^{-j\frac{2\pi kn}{N}}, \quad k = 0, 1, \ldots, N-1
\end{equation}
\end{definition}

The inverse DFT reconstructs the time-domain signal:
\begin{equation}
x[n] = \frac{1}{N} \sum_{k=0}^{N-1} X[k] \cdot e^{j\frac{2\pi kn}{N}}, \quad n = 0, 1, \ldots, N-1
\end{equation}

Using Euler's formula $e^{j\theta} = \cos(\theta) + j\sin(\theta)$, we can express the DFT in terms of real operations:
\begin{align}
\text{Re}(X[k]) &= \sum_{n=0}^{N-1} x[n] \cos\left(\frac{2\pi kn}{N}\right) \\
\text{Im}(X[k]) &= -\sum_{n=0}^{N-1} x[n] \sin\left(\frac{2\pi kn}{N}\right)
\end{align}

The magnitude and phase spectra are computed as:
\begin{equation}
|X[k]| = \sqrt{\text{Re}(X[k])^2 + \text{Im}(X[k])^2}, \quad \angle X[k] = \arctan\left(\frac{\text{Im}(X[k])}{\text{Re}(X[k])}\right)
\end{equation}

\subsection{Fast Fourier Transform}

The Cooley-Tukey FFT algorithm reduces the computational complexity of the DFT from $O(N^2)$ to $O(N \log N)$ through recursive decomposition:

\begin{algorithm}
\caption{Radix-2 Decimation-in-Time FFT}
\begin{algorithmic}[1]
\REQUIRE Signal $x[0:N-1]$ where $N = 2^m$
\ENSURE DFT coefficients $X[0:N-1]$
\IF{$N = 1$}
    \RETURN $x[0]$
\ENDIF
\STATE $X_{\text{even}} \gets \text{FFT}(x[0], x[2], \ldots, x[N-2])$
\STATE $X_{\text{odd}} \gets \text{FFT}(x[1], x[3], \ldots, x[N-1])$
\FOR{$k = 0$ to $N/2 - 1$}
    \STATE $W_N^k \gets e^{-j2\pi k/N}$ \COMMENT{Twiddle factor}
    \STATE $X[k] \gets X_{\text{even}}[k] + W_N^k \cdot X_{\text{odd}}[k]$
    \STATE $X[k + N/2] \gets X_{\text{even}}[k] - W_N^k \cdot X_{\text{odd}}[k]$
\ENDFOR
\RETURN $X$
\end{algorithmic}
\end{algorithm}

\subsection{Short-Time Fourier Transform}

For non-stationary signals like speech, the Short-Time Fourier Transform (STFT) provides time-frequency analysis by applying the DFT to windowed segments:

\begin{equation}
X[m, k] = \sum_{n=0}^{N-1} x[n + mH] \cdot w[n] \cdot e^{-j\frac{2\pi kn}{N}}
\end{equation}

where $m$ is the frame index, $H$ is the hop size (frame shift), and $w[n]$ is the analysis window. Common window functions include:

\textbf{Hamming Window:}
\begin{equation}
w[n] = 0.54 - 0.46 \cos\left(\frac{2\pi n}{N-1}\right)
\end{equation}

\textbf{Hann Window:}
\begin{equation}
w[n] = 0.5 \left(1 - \cos\left(\frac{2\pi n}{N-1}\right)\right)
\end{equation}

\textbf{Blackman Window:}
\begin{equation}
w[n] = 0.42 - 0.5 \cos\left(\frac{2\pi n}{N-1}\right) + 0.08 \cos\left(\frac{4\pi n}{N-1}\right)
\end{equation}

\subsection{Mel-Frequency Analysis}

Human auditory perception is approximately logarithmic in frequency. The mel scale converts linear frequency to perceptual frequency:

\begin{equation}
m = 2595 \log_{10}\left(1 + \frac{f}{700}\right)
\end{equation}

The inverse transformation is:
\begin{equation}
f = 700 \left(10^{m/2595} - 1\right)
\end{equation}

Mel filterbanks are triangular filters spaced uniformly on the mel scale. For $M$ filters spanning frequencies $f_{\min}$ to $f_{\max}$:

\begin{equation}
H_m[k] = \begin{cases}
0 & k < f[m-1] \\
\frac{k - f[m-1]}{f[m] - f[m-1]} & f[m-1] \leq k < f[m] \\
\frac{f[m+1] - k}{f[m+1] - f[m]} & f[m] \leq k < f[m+1] \\
0 & k \geq f[m+1]
\end{cases}
\end{equation}

The mel-frequency spectrogram is computed as:
\begin{equation}
\text{MelSpec}[m, t] = \log\left(\sum_{k} H_m[k] |X[t, k]|^2 + \epsilon\right)
\end{equation}

\section{Digital Filter Design}

\subsection{FIR Filters}

Finite Impulse Response (FIR) filters implement the convolution operation:
\begin{equation}
y[n] = \sum_{k=0}^{M-1} h[k] \cdot x[n-k]
\end{equation}

where $h[k]$ are the filter coefficients and $M$ is the filter order. The frequency response is:
\begin{equation}
H(e^{j\omega}) = \sum_{k=0}^{M-1} h[k] e^{-j\omega k}
\end{equation}

For lowpass filter design using the windowed sinc method, the ideal impulse response is:
\begin{equation}
h_{\text{ideal}}[n] = \frac{2f_c}{f_s} \text{sinc}\left(\frac{2f_c(n - M/2)}{f_s}\right)
\end{equation}

where $\text{sinc}(x) = \sin(\pi x)/(\pi x)$. Windowing reduces the Gibbs phenomenon:
\begin{equation}
h[n] = h_{\text{ideal}}[n] \cdot w[n]
\end{equation}

\subsection{IIR Filters}

Infinite Impulse Response (IIR) filters include feedback terms:
\begin{equation}
y[n] = \sum_{k=0}^{M} b_k x[n-k] - \sum_{k=1}^{N} a_k y[n-k]
\end{equation}

The transfer function in the z-domain is:
\begin{equation}
H(z) = \frac{\sum_{k=0}^{M} b_k z^{-k}}{1 + \sum_{k=1}^{N} a_k z^{-k}} = \frac{B(z)}{A(z)}
\end{equation}

\textbf{Butterworth Filter:} Maximally flat magnitude response:
\begin{equation}
|H(j\omega)|^2 = \frac{1}{1 + \left(\omega/\omega_c\right)^{2N}}
\end{equation}

The poles of an $N$th-order Butterworth filter lie on a circle in the s-plane:
\begin{equation}
s_k = \omega_c e^{j\pi(2k+N+1)/(2N)}, \quad k = 0, 1, \ldots, N-1
\end{equation}

Digital filter coefficients are obtained via the bilinear transform:
\begin{equation}
s = \frac{2}{T} \cdot \frac{1 - z^{-1}}{1 + z^{-1}}
\end{equation}

\subsection{Adaptive Filters}

Adaptive filters adjust their coefficients to minimize an error criterion. The Normalized Least Mean Squares (NLMS) algorithm updates coefficients as:
\begin{equation}
\mathbf{w}[n+1] = \mathbf{w}[n] + \frac{\mu \cdot e[n] \cdot \mathbf{x}[n]}{\|\mathbf{x}[n]\|^2 + \epsilon}
\end{equation}

where $e[n] = d[n] - \mathbf{w}^T[n]\mathbf{x}[n]$ is the error signal, $\mu$ is the step size, and $\epsilon$ prevents division by zero.

The Recursive Least Squares (RLS) algorithm provides faster convergence:
\begin{align}
\mathbf{k}[n] &= \frac{\mathbf{P}[n-1]\mathbf{x}[n]}{\lambda + \mathbf{x}^T[n]\mathbf{P}[n-1]\mathbf{x}[n]} \\
e[n] &= d[n] - \mathbf{w}^T[n-1]\mathbf{x}[n] \\
\mathbf{w}[n] &= \mathbf{w}[n-1] + \mathbf{k}[n] e[n] \\
\mathbf{P}[n] &= \frac{1}{\lambda}\left(\mathbf{P}[n-1] - \mathbf{k}[n]\mathbf{x}^T[n]\mathbf{P}[n-1]\right)
\end{align}

where $\lambda \in (0, 1]$ is the forgetting factor and $\mathbf{P}$ is the inverse correlation matrix.

\section{Transformer Architecture}

\subsection{Self-Attention Mechanism}

The scaled dot-product attention mechanism computes weighted combinations of values based on query-key similarities:

\begin{equation}
\text{Attention}(Q, K, V) = \text{softmax}\left(\frac{QK^T}{\sqrt{d_k}}\right)V
\end{equation}

where $Q \in \mathbb{R}^{n \times d_k}$, $K \in \mathbb{R}^{m \times d_k}$, and $V \in \mathbb{R}^{m \times d_v}$ are the query, key, and value matrices respectively. The scaling factor $\sqrt{d_k}$ prevents the dot products from growing too large for stable softmax computation.

Multi-head attention projects inputs into multiple subspaces:
\begin{align}
\text{MultiHead}(Q, K, V) &= \text{Concat}(\text{head}_1, \ldots, \text{head}_h)W^O \\
\text{head}_i &= \text{Attention}(QW_i^Q, KW_i^K, VW_i^V)
\end{align}

where $W_i^Q \in \mathbb{R}^{d_{\text{model}} \times d_k}$, $W_i^K \in \mathbb{R}^{d_{\text{model}} \times d_k}$, $W_i^V \in \mathbb{R}^{d_{\text{model}} \times d_v}$, and $W^O \in \mathbb{R}^{hd_v \times d_{\text{model}}}$ are learned projection matrices.

\subsection{Positional Encoding}

Transformers require explicit position information since attention is permutation-invariant. The original sinusoidal positional encoding is:
\begin{align}
PE_{(pos, 2i)} &= \sin\left(\frac{pos}{10000^{2i/d_{\text{model}}}}\right) \\
PE_{(pos, 2i+1)} &= \cos\left(\frac{pos}{10000^{2i/d_{\text{model}}}}\right)
\end{align}

\subsection{Rotary Position Embedding (RoPE)}

RoPE encodes position through rotation in the complex plane, enabling relative position awareness \cite{su2021roformer}:

\begin{equation}
f_q(\mathbf{x}_m, m) = R_{\Theta,m}^d W_q \mathbf{x}_m
\end{equation}

where the rotation matrix for position $m$ is:
\begin{equation}
R_{\Theta,m}^d = \begin{pmatrix}
\cos(m\theta_1) & -\sin(m\theta_1) & & \\
\sin(m\theta_1) & \cos(m\theta_1) & & \\
& & \ddots & \\
& & & \cos(m\theta_{d/2}) & -\sin(m\theta_{d/2}) \\
& & & \sin(m\theta_{d/2}) & \cos(m\theta_{d/2})
\end{pmatrix}
\end{equation}

with $\theta_i = 10000^{-2(i-1)/d}$. The key property is that the attention score depends on relative position:
\begin{equation}
\langle R_{\Theta,m}^d \mathbf{q}, R_{\Theta,n}^d \mathbf{k} \rangle = \langle \mathbf{q}, R_{\Theta,n-m}^d \mathbf{k} \rangle
\end{equation}

\subsection{SwiGLU Activation}

The SwiGLU activation function enhances feedforward networks \cite{shazeer2020glu}:
\begin{equation}
\text{SwiGLU}(x) = \text{Swish}(xW_1) \otimes (xW_2)
\end{equation}

where $\text{Swish}(x) = x \cdot \sigma(x)$ and $\sigma$ is the sigmoid function. This replaces the standard ReLU-based FFN:
\begin{equation}
\text{FFN}_{\text{SwiGLU}}(x) = (\text{Swish}(xW_1) \otimes xW_2)W_3
\end{equation}

\subsection{Conformer Architecture}

The Conformer block combines self-attention with convolution for speech processing \cite{gulati2020conformer}:
\begin{align}
\tilde{x} &= x + \frac{1}{2}\text{FFN}(x) \\
x' &= \tilde{x} + \text{MHSA}(\tilde{x}) \\
x'' &= x' + \text{Conv}(x') \\
y &= \text{LayerNorm}(x'' + \frac{1}{2}\text{FFN}(x''))
\end{align}

The convolution module uses depthwise separable convolution:
\begin{equation}
\text{Conv}(x) = \text{BatchNorm}(\text{GLU}(\text{PointwiseConv}(\text{DepthwiseConv}(\text{LayerNorm}(x)))))
\end{equation}

\section{Speech Recognition Theory}

\subsection{Connectionist Temporal Classification (CTC)}

CTC enables sequence-to-sequence training without explicit alignment \cite{graves2006ctc}. Given input sequence $\mathbf{x}$ of length $T$ and target sequence $\mathbf{y}$ of length $U$ where $U \leq T$, CTC introduces a blank symbol $\epsilon$ and defines the probability:

\begin{equation}
P(\mathbf{y}|\mathbf{x}) = \sum_{\pi \in \mathcal{B}^{-1}(\mathbf{y})} P(\pi|\mathbf{x})
\end{equation}

where $\mathcal{B}^{-1}(\mathbf{y})$ is the set of all paths that collapse to $\mathbf{y}$ after removing blanks and repeated characters.

The forward variable $\alpha_t(s)$ represents the probability of outputting the first $s$ symbols of $\mathbf{y}'$ (with blanks) at time $t$:
\begin{equation}
\alpha_t(s) = \sum_{\pi_{1:t}: \mathcal{B}(\pi_{1:t}) = \mathbf{y}'_{1:s}} \prod_{t'=1}^{t} P(\pi_{t'}|\mathbf{x})
\end{equation}

The recurrence relations are:
\begin{equation}
\alpha_t(s) = \begin{cases}
(\alpha_{t-1}(s) + \alpha_{t-1}(s-1)) \cdot p_t(y'_s) & \text{if } y'_s = \epsilon \text{ or } y'_s = y'_{s-2} \\
(\alpha_{t-1}(s) + \alpha_{t-1}(s-1) + \alpha_{t-1}(s-2)) \cdot p_t(y'_s) & \text{otherwise}
\end{cases}
\end{equation}

The CTC loss is:
\begin{equation}
\mathcal{L}_{\text{CTC}} = -\log P(\mathbf{y}|\mathbf{x}) = -\log(\alpha_T(|\mathbf{y}'|) + \alpha_T(|\mathbf{y}'|-1))
\end{equation}

\subsection{Hybrid CTC-Attention Loss}

Modern speech recognition systems combine CTC with attention-based losses:
\begin{equation}
\mathcal{L} = \lambda \mathcal{L}_{\text{CTC}} + (1-\lambda) \mathcal{L}_{\text{CE}}
\end{equation}

where $\mathcal{L}_{\text{CE}}$ is the cross-entropy loss for autoregressive token prediction and $\lambda \in [0, 1]$ balances the objectives.

\section{Retrieval-Augmented Generation}

\subsection{Dense Retrieval}

Dense retrieval encodes queries and documents into continuous vector spaces using neural encoders:
\begin{align}
\mathbf{q} &= E_q(\text{query}) \in \mathbb{R}^d \\
\mathbf{d} &= E_d(\text{document}) \in \mathbb{R}^d
\end{align}

Relevance is computed via similarity:
\begin{equation}
\text{sim}(\mathbf{q}, \mathbf{d}) = \frac{\mathbf{q}^T \mathbf{d}}{\|\mathbf{q}\| \|\mathbf{d}\|}
\end{equation}

\subsection{BM25 Sparse Retrieval}

BM25 (Best Matching 25) computes lexical relevance \cite{karpukhin2020dpr}:
\begin{equation}
\text{BM25}(D, Q) = \sum_{i=1}^{n} \text{IDF}(q_i) \cdot \frac{f(q_i, D) \cdot (k_1 + 1)}{f(q_i, D) + k_1 \cdot (1 - b + b \cdot |D|/\text{avgdl})}
\end{equation}

where:
\begin{equation}
\text{IDF}(q_i) = \log\left(\frac{N - n(q_i) + 0.5}{n(q_i) + 0.5}\right)
\end{equation}

with $N$ = total documents, $n(q_i)$ = documents containing term $q_i$, $f(q_i, D)$ = term frequency, $|D|$ = document length, avgdl = average document length, $k_1 \approx 1.5$, and $b \approx 0.75$.

\subsection{Reciprocal Rank Fusion}

RRF combines multiple ranked lists without score normalization:
\begin{equation}
\text{RRF}(d) = \sum_{r \in R} \frac{1}{k + \text{rank}_r(d)}
\end{equation}

where $R$ is the set of retrievers, $\text{rank}_r(d)$ is the rank of document $d$ in retriever $r$, and $k=60$ is a constant.

\subsection{Cross-Encoder Reranking}

Cross-encoders compute relevance by jointly encoding query-document pairs:
\begin{equation}
\text{score}(q, d) = \text{sigmoid}(\text{Linear}(\text{Transformer}([CLS]; q; [SEP]; d)))
\end{equation}

This captures fine-grained query-document interactions at the cost of increased inference latency.

\subsection{RAGAS Evaluation Framework}

The RAGAS framework provides metrics for RAG system evaluation \cite{es2023ragas}:

\textbf{Faithfulness:}
\begin{equation}
\text{Faithfulness} = \frac{|\text{Claims}_{\text{supported}}|}{|\text{Claims}_{\text{total}}|}
\end{equation}

\textbf{Answer Relevancy:}
\begin{equation}
\text{Relevancy} = \frac{1}{N} \sum_{i=1}^{N} \text{sim}(q, q_i^{\text{generated}})
\end{equation}

\textbf{Context Precision:}
\begin{equation}
\text{Precision@K} = \frac{\sum_{k=1}^{K} (\text{Precision@k} \times v_k)}{\text{Total relevant in top K}}
\end{equation}

where $v_k = 1$ if context at position $k$ is relevant.

% =============================================================================
% CHAPTER 3: METHODOLOGY
% =============================================================================
\chapter{Methodology}

This chapter presents the detailed methodology for each of the five system components. We describe architectural decisions, mathematical formulations, training procedures, and implementation details.

\section{VoxFormer: Speech-to-Text Transformer}

\subsection{System Overview}

VoxFormer is a custom encoder-decoder architecture designed for robust speech recognition in technical domains. The architecture consists of three main components:

\begin{enumerate}
    \item \textbf{WavLM Acoustic Encoder}: Pre-trained self-supervised speech representation model (95M parameters)
    \item \textbf{Zipformer Temporal Encoder}: Modified Conformer blocks with downsampling (47M parameters)
    \item \textbf{Transformer Decoder}: Autoregressive token prediction with cross-attention
\end{enumerate}

\begin{figure}[htbp]
\centering
\fbox{\parbox{0.9\textwidth}{\centering
\vspace{2cm}
\textbf{[FIGURE PLACEHOLDER]}\\
VoxFormer Architecture Diagram\\
Showing WavLM encoder, Zipformer blocks, and Transformer decoder\\
\vspace{2cm}
}}
\caption{Complete VoxFormer architecture with component dimensions and connections.}
\label{fig:voxformer-architecture}
\end{figure}

\subsection{Audio Frontend}

The audio frontend processes raw waveforms into acoustic features suitable for the neural encoder.

\subsubsection{Pre-emphasis Filtering}

A first-order high-pass filter compensates for the natural spectral tilt of speech:
\begin{equation}
y[n] = x[n] - \alpha x[n-1], \quad \alpha = 0.97
\end{equation}

This boosts high-frequency components containing important consonant information.

\subsubsection{STFT Configuration}

The Short-Time Fourier Transform parameters are:
\begin{itemize}
    \item Window size: 25ms (400 samples at 16kHz)
    \item Hop size: 10ms (160 samples)
    \item FFT size: 512 points
    \item Window function: Hann window
\end{itemize}

\subsubsection{Mel Filterbank}

An 80-channel mel filterbank is applied to the power spectrum:
\begin{equation}
\text{MelSpec}[m, t] = \log\left(\sum_{k=f_{\text{low}}[m]}^{f_{\text{high}}[m]} H_m[k] |X[t, k]|^2 + 10^{-10}\right)
\end{equation}

The filterbank spans 20Hz to 8000Hz with triangular filters spaced uniformly on the mel scale.

\subsubsection{Global Mean-Variance Normalization}

Features are normalized using corpus statistics:
\begin{equation}
\hat{x} = \frac{x - \mu}{\sigma + \epsilon}
\end{equation}

where $\mu$ and $\sigma$ are computed over the training set.

\subsection{WavLM Encoder}

WavLM is a self-supervised pre-trained model that learns speech representations from unlabeled audio. Key specifications:

\begin{table}[htbp]
\centering
\caption{WavLM Encoder Specifications}
\label{tab:wavlm-specs}
\begin{tabular}{ll}
\toprule
\textbf{Parameter} & \textbf{Value} \\
\midrule
Architecture & Transformer \\
Parameters & 95M \\
Hidden dimension & 768 \\
Attention heads & 12 \\
Layers & 12 \\
Input & Raw waveform (16kHz) \\
Output & Frame-level features (50Hz) \\
Pre-training data & 60,000 hours \\
\bottomrule
\end{tabular}
\end{table}

The WavLM output provides rich acoustic representations that capture both phonetic content and speaker characteristics.

\subsection{Zipformer Temporal Encoder}

The Zipformer encoder processes the acoustic features through 6 Conformer-style blocks with progressive downsampling.

\subsubsection{Block Configuration}

Each Zipformer block follows the Conformer architecture with modifications:

\begin{align}
\tilde{x} &= x + 0.5 \cdot \text{FFN}_{\text{SwiGLU}}(\text{LayerNorm}(x)) \\
x' &= \tilde{x} + \text{MHSA}_{\text{RoPE}}(\text{LayerNorm}(\tilde{x})) \\
x'' &= x' + \text{ConvModule}(\text{LayerNorm}(x')) \\
y &= \text{LayerNorm}(x'' + 0.5 \cdot \text{FFN}_{\text{SwiGLU}}(\text{LayerNorm}(x'')))
\end{align}

\subsubsection{Multi-Head Self-Attention with RoPE}

For each attention head, queries and keys are rotated based on position:
\begin{equation}
\text{Attention}(Q, K, V) = \text{softmax}\left(\frac{(R_\Theta^m Q)(R_\Theta^n K)^T}{\sqrt{d_k}}\right)V
\end{equation}

The rotation matrices encode relative positions efficiently.

\subsubsection{Convolution Module}

The convolution module captures local patterns:
\begin{equation}
\text{ConvModule}(x) = \text{GLU}(\text{DepthwiseConv}_{31}(\text{PointwiseConv}(x)))
\end{equation}

with kernel size 31 for approximately 300ms context.

\subsubsection{Downsampling Schedule}

\begin{table}[htbp]
\centering
\caption{Zipformer Downsampling Configuration}
\label{tab:zipformer-downsample}
\begin{tabular}{cccc}
\toprule
\textbf{Block} & \textbf{Input Rate} & \textbf{Output Rate} & \textbf{Downsample Factor} \\
\midrule
1 & 50 Hz & 50 Hz & 1x \\
2 & 50 Hz & 25 Hz & 2x \\
3 & 25 Hz & 25 Hz & 1x \\
4 & 25 Hz & 12.5 Hz & 2x \\
5 & 12.5 Hz & 12.5 Hz & 1x \\
6 & 12.5 Hz & 12.5 Hz & 1x \\
\bottomrule
\end{tabular}
\end{table}

\subsection{Transformer Decoder}

The decoder generates text tokens autoregressively with cross-attention to encoder outputs.

\subsubsection{Architecture}

\begin{itemize}
    \item Layers: 6
    \item Hidden dimension: 512
    \item Attention heads: 8
    \item Vocabulary: 5,000 BPE tokens
\end{itemize}

\subsubsection{Decoding}

\begin{equation}
P(y_t | y_{<t}, x) = \text{softmax}(W_o \cdot \text{DecoderBlock}(\text{Embed}(y_{<t}), \text{Encoder}(x)))
\end{equation}

Beam search with beam width 10 and length penalty 0.6 is used during inference.

\subsection{Training Objective}

The hybrid loss combines CTC and cross-entropy:
\begin{equation}
\mathcal{L} = 0.3 \mathcal{L}_{\text{CTC}} + 0.7 \mathcal{L}_{\text{CE}}
\end{equation}

\subsubsection{CTC Loss Computation}

CTC loss is computed on the encoder output with an additional linear projection:
\begin{equation}
\mathcal{L}_{\text{CTC}} = -\log P_{\text{CTC}}(\mathbf{y} | \mathbf{x})
\end{equation}

\subsubsection{Cross-Entropy Loss}

Teacher-forced cross-entropy on decoder outputs:
\begin{equation}
\mathcal{L}_{\text{CE}} = -\sum_{t=1}^{T} \log P(y_t | y_{<t}, \mathbf{x})
\end{equation}

\subsection{Training Strategy}

\subsubsection{3-Stage Curriculum Learning}

\begin{table}[htbp]
\centering
\caption{VoxFormer Training Stages}
\label{tab:voxformer-training}
\begin{tabular}{cccccc}
\toprule
\textbf{Stage} & \textbf{Epochs} & \textbf{Data} & \textbf{LR} & \textbf{Batch Size} & \textbf{Frozen} \\
\midrule
1 & 20 & clean-100 & 1e-3 & 32 & WavLM \\
2 & 30 & clean-360 & 5e-4 & 64 & None \\
3 & 20 & full-960 & 1e-4 & 128 & None \\
\bottomrule
\end{tabular}
\end{table}

\subsubsection{Optimizer Configuration}

\begin{itemize}
    \item Optimizer: AdamW
    \item Weight decay: 0.01
    \item $\beta_1 = 0.9$, $\beta_2 = 0.98$
    \item Warmup: 10,000 steps
    \item Scheduler: Inverse square root decay
\end{itemize}

\subsubsection{Data Augmentation}

\begin{itemize}
    \item SpecAugment: 2 frequency masks (width 27), 2 time masks (width 100)
    \item Speed perturbation: 0.9x, 1.0x, 1.1x
    \item Additive noise: SNR 10-30dB
\end{itemize}

\begin{figure}[htbp]
\centering
\fbox{\parbox{0.9\textwidth}{\centering
\vspace{2cm}
\textbf{[FIGURE PLACEHOLDER]}\\
Training Loss Curves\\
Showing CTC loss, CE loss, and WER during training\\
\vspace{2cm}
}}
\caption{VoxFormer training convergence over 70 epochs.}
\label{fig:training-curves}
\end{figure}

\section{Advanced RAG System}

\subsection{System Architecture}

The RAG system implements a 7-layer agentic architecture for knowledge-intensive question answering about 3D game development.

\begin{figure}[htbp]
\centering
\fbox{\parbox{0.9\textwidth}{\centering
\vspace{2cm}
\textbf{[FIGURE PLACEHOLDER]}\\
RAG System Architecture\\
7-layer pipeline from query to response\\
\vspace{2cm}
}}
\caption{Advanced RAG system architecture with hybrid retrieval and agentic validation.}
\label{fig:rag-architecture}
\end{figure}

\subsection{Document Processing Pipeline}

\subsubsection{Document Schema}

Documents are stored with comprehensive metadata:

\begin{lstlisting}[language=SQL, caption=Document Schema]
CREATE TABLE documents (
    id BIGSERIAL PRIMARY KEY,
    content TEXT NOT NULL,
    title VARCHAR(500),
    embedding vector(4096),
    source VARCHAR(50),
    blender_version VARCHAR(10),
    category VARCHAR(50),
    subcategory VARCHAR(50),
    language VARCHAR(20),
    is_code BOOLEAN,
    priority FLOAT DEFAULT 0.5,
    tsv tsvector GENERATED ALWAYS AS (
        to_tsvector('english', content)
    ) STORED
);
\end{lstlisting}

\subsubsection{Chunking Strategy}

Documents are split using semantic chunking:
\begin{itemize}
    \item Target chunk size: 512 tokens
    \item Overlap: 64 tokens
    \item Split on sentence boundaries
    \item Preserve code blocks intact
\end{itemize}

\subsection{Embedding Model: BGE-M3}

The BGE-M3 model generates dense embeddings:

\begin{table}[htbp]
\centering
\caption{BGE-M3 Embedding Model Specifications}
\label{tab:bge-specs}
\begin{tabular}{ll}
\toprule
\textbf{Parameter} & \textbf{Value} \\
\midrule
Dimension & 4,096 \\
Max sequence length & 8,192 tokens \\
Model size & 2.3GB \\
Inference latency & 150ms per 1K documents \\
Quantization & INT8 supported \\
\bottomrule
\end{tabular}
\end{table}

\subsection{Hybrid Retrieval}

\subsubsection{Dense Retrieval with HNSW}

Vector search uses HNSW (Hierarchical Navigable Small World) indexing:

\begin{equation}
\text{HNSW Parameters: } m=16, \text{ef\_construction}=200, \text{ef}=100
\end{equation}

Complexity: $O(\log N)$ for search with $O(N \cdot m)$ space overhead.

\begin{lstlisting}[language=SQL, caption=HNSW Index Creation]
CREATE INDEX documents_embedding_idx
ON documents
USING hnsw (embedding vector_cosine_ops)
WITH (m=16, ef_construction=200);
\end{lstlisting}

\subsubsection{Sparse Retrieval with BM25}

Full-text search via PostgreSQL:

\begin{lstlisting}[language=SQL, caption=BM25 Search Query]
SELECT id, content,
    ts_rank(tsv, plainto_tsquery('english', $1)) as score
FROM documents
WHERE tsv @@ plainto_tsquery('english', $1)
ORDER BY score DESC
LIMIT 100;
\end{lstlisting}

\subsubsection{Reciprocal Rank Fusion}

Results from dense and sparse retrieval are fused:

\begin{algorithm}
\caption{Reciprocal Rank Fusion}
\begin{algorithmic}[1]
\REQUIRE Dense results $D = [(d_1, s_1), \ldots]$, Sparse results $S = [(d_1, s_1), \ldots]$
\ENSURE Fused ranking
\STATE $k \gets 60$
\STATE scores $\gets \{\}$
\FOR{$(d, \_)$ at rank $r$ in $D$}
    \STATE scores$[d]$ += $1/(k + r)$
\ENDFOR
\FOR{$(d, \_)$ at rank $r$ in $S$}
    \STATE scores$[d]$ += $1/(k + r)$
\ENDFOR
\RETURN sorted(scores, by=value, descending)
\end{algorithmic}
\end{algorithm}

\subsection{Cross-Encoder Reranking}

The MiniLM cross-encoder reranks top-50 candidates to select top-10:

\begin{table}[htbp]
\centering
\caption{Cross-Encoder Specifications}
\label{tab:reranker-specs}
\begin{tabular}{ll}
\toprule
\textbf{Parameter} & \textbf{Value} \\
\midrule
Model & cross-encoder/ms-marco-MiniLM-L-6-v2 \\
Parameters & 66M \\
Latency & 15ms per candidate \\
Accuracy (MS MARCO) & 0.89 \\
\bottomrule
\end{tabular}
\end{table}

\subsection{Agentic Query Transformation}

\subsubsection{Query Analysis}

The query analyzer extracts:
\begin{itemize}
    \item Intent classification (e.g., 'rotate\_objects', 'select\_faces')
    \item Entity extraction (objects, operations, modifiers)
    \item Scope identification (all, selected, specific)
    \item Query variations for retrieval diversity
\end{itemize}

\subsubsection{Multi-Hop Decomposition}

Complex queries are decomposed into sequential sub-queries:

\begin{lstlisting}[caption=Query Decomposition Example]
Input: "Rotate all faces and apply smooth shading"

Sub-queries:
1. "How to select all faces in Blender"
2. "How to rotate selected faces"
3. "How to apply smooth shading"
\end{lstlisting}

\subsection{Validation Loop}

The agentic validation loop ensures answer quality:

\begin{algorithm}
\caption{Agentic RAG with Validation}
\begin{algorithmic}[1]
\REQUIRE Query $q$, max\_retries $= 3$
\ENSURE Answer $a$, metadata
\FOR{attempt $= 1$ to max\_retries}
    \STATE context $\gets$ RetrieveAndRerank($q$)
    \STATE answer $\gets$ GenerateAnswer($q$, context)
    \STATE validation $\gets$ ValidateAnswer($q$, context, answer)
    \IF{validation.is\_valid}
        \RETURN answer, metadata
    \ENDIF
    \STATE $q \gets$ RewriteQuery($q$, validation.issues)
\ENDFOR
\RETURN answer, metadata \COMMENT{Best effort}
\end{algorithmic}
\end{algorithm}

\subsubsection{Validation Checks}

\begin{enumerate}
    \item \textbf{Syntax Validation}: Python code compilation
    \item \textbf{API Compatibility}: Blender version checking
    \item \textbf{Grounding Score}: Context-answer alignment
    \item \textbf{Hallucination Detection}: Ungrounded claims identification
\end{enumerate}

\subsection{RAGAS Evaluation}

Continuous quality monitoring uses RAGAS metrics:

\begin{equation}
\text{RAGAS Score} = 0.25 \cdot F + 0.25 \cdot R + 0.25 \cdot P + 0.25 \cdot C
\end{equation}

where $F$ = Faithfulness, $R$ = Answer Relevancy, $P$ = Context Precision, $C$ = Context Recall.

\section{Text-to-Speech and Lip Synchronization}

\subsection{TTS Pipeline}

\subsubsection{ElevenLabs Integration}

The system uses ElevenLabs Flash v2.5 for low-latency speech synthesis:

\begin{table}[htbp]
\centering
\caption{ElevenLabs TTS Specifications}
\label{tab:elevenlabs-specs}
\begin{tabular}{ll}
\toprule
\textbf{Parameter} & \textbf{Value} \\
\midrule
Model & Flash v2.5 \\
Time to First Byte & 75ms \\
MOS Score & 4.14 \\
Sample Rate & 44.1kHz \\
Streaming & Yes \\
Multilingual & 29 languages \\
\bottomrule
\end{tabular}
\end{table}

\subsubsection{Phoneme-to-Viseme Mapping}

Speech phonemes are mapped to visual mouth shapes (visemes):

\begin{table}[htbp]
\centering
\caption{Phoneme-to-Viseme Mapping}
\label{tab:viseme-mapping}
\begin{tabular}{lll}
\toprule
\textbf{Viseme} & \textbf{Phonemes} & \textbf{Description} \\
\midrule
V0 & Silence & Mouth closed \\
V1 & /p/, /b/, /m/ & Lips together \\
V2 & /f/, /v/ & Lower lip to upper teeth \\
V3 & /th/ & Tongue between teeth \\
V4 & /t/, /d/, /n/, /l/ & Tongue to alveolar ridge \\
V5 & /k/, /g/, /ng/ & Back of tongue raised \\
V6 & /s/, /z/ & Teeth together \\
V7 & /sh/, /ch/, /j/ & Lips rounded, teeth apart \\
V8 & /r/ & Lips slightly rounded \\
V9 & /aa/, /ae/ & Wide open \\
V10 & /ow/, /uw/ & Rounded, small opening \\
V11 & /eh/, /ah/ & Medium open \\
V12 & /ih/, /iy/ & Slightly open, spread \\
V13 & /aw/ & Wide to rounded \\
V14 & /oy/ & Rounded to spread \\
\bottomrule
\end{tabular}
\end{table}

\subsection{Lip Synchronization}

\subsubsection{SadTalker Architecture}

SadTalker generates realistic talking head videos from audio and a single image \cite{zhang2023sadtalker}:

\begin{enumerate}
    \item \textbf{Audio-to-3DMM}: ExpNet predicts 3D Morphable Model coefficients from audio
    \item \textbf{Pose Generation}: PoseVAE generates natural head motions
    \item \textbf{Face Rendering}: 3D warping with face rendering network
\end{enumerate}

The 3DMM representation decomposes facial appearance as:
\begin{equation}
S = \bar{S} + \sum_{i=1}^{n} \alpha_i S_i^{\text{id}} + \sum_{j=1}^{m} \beta_j S_j^{\text{exp}}
\end{equation}

where $\bar{S}$ is the mean face, $S_i^{\text{id}}$ are identity bases, and $S_j^{\text{exp}}$ are expression bases.

\subsubsection{MuseTalk Latent Space Inpainting}

MuseTalk provides real-time lip synchronization through latent space manipulation \cite{zhang2024musetalk}:

\begin{figure}[htbp]
\centering
\fbox{\parbox{0.9\textwidth}{\centering
\vspace{2cm}
\textbf{[FIGURE PLACEHOLDER]}\\
MuseTalk Architecture\\
Audio encoder, latent inpainting, and decoder\\
\vspace{2cm}
}}
\caption{MuseTalk latent space lip synchronization pipeline.}
\label{fig:musetalk-architecture}
\end{figure}

\subsubsection{Loss Functions}

The lip-sync training uses multiple loss terms:

\textbf{Perceptual Loss:}
\begin{equation}
\mathcal{L}_{\text{perceptual}} = \sum_{l} \|\phi_l(G(z)) - \phi_l(y)\|_2^2
\end{equation}

where $\phi_l$ are VGG-19 feature maps.

\textbf{Sync Loss:}
\begin{equation}
\mathcal{L}_{\text{sync}} = 1 - \cos(\text{Embed}_{\text{lip}}(v), \text{Embed}_{\text{audio}}(a))
\end{equation}

\textbf{GAN Loss:}
\begin{equation}
\mathcal{L}_{\text{GAN}} = \mathbb{E}[\log D(y)] + \mathbb{E}[\log(1 - D(G(z)))]
\end{equation}

\textbf{Total Loss:}
\begin{equation}
\mathcal{L}_{\text{total}} = \lambda_1 \mathcal{L}_{\text{perceptual}} + \lambda_2 \mathcal{L}_{\text{sync}} + \lambda_3 \mathcal{L}_{\text{GAN}} + \lambda_4 \mathcal{L}_{\text{L1}}
\end{equation}

\section{DSP Voice Isolation Pipeline}

\subsection{6-Stage Pipeline Architecture}

The voice isolation pipeline implements a comprehensive signal processing chain:

\begin{figure}[htbp]
\centering
\fbox{\parbox{0.9\textwidth}{\centering
\vspace{3cm}
\textbf{[FIGURE PLACEHOLDER]}\\
6-Stage DSP Pipeline Architecture\\
Signal Conditioning → VAD → Noise Estimation → Enhancement → Separation → Dereverberation\\
\vspace{2cm}
}}
\caption{Complete DSP voice isolation pipeline with all processing stages.}
\label{fig:dsp-pipeline}
\end{figure}

\subsection{Stage 1: Signal Conditioning}

\subsubsection{DC Offset Removal}

Remove DC component:
\begin{equation}
y[n] = x[n] - \frac{1}{N}\sum_{k=0}^{N-1} x[k]
\end{equation}

\subsubsection{Pre-emphasis Filter}

Boost high frequencies:
\begin{equation}
y[n] = x[n] - 0.97 \cdot x[n-1]
\end{equation}

The transfer function is $H(z) = 1 - 0.97z^{-1}$.

\subsubsection{Dithering}

Add small noise to decorrelate quantization error:
\begin{equation}
x_{\text{dithered}}[n] = x[n] + d[n], \quad d[n] \sim \text{Triangular}(-\Delta, \Delta)
\end{equation}

\subsection{Stage 2: Voice Activity Detection}

\subsubsection{Energy-Based VAD}

Frame energy is computed as:
\begin{equation}
E[m] = \sum_{n=0}^{N-1} |x[mH + n]|^2
\end{equation}

Log energy in dB:
\begin{equation}
E_{\text{dB}}[m] = 10 \log_{10}(E[m] + \epsilon)
\end{equation}

Speech decision with adaptive threshold:
\begin{equation}
\text{VAD}[m] = \begin{cases}
1 & \text{if } E_{\text{dB}}[m] > \text{NoiseFloor} + 15\text{dB} \\
0 & \text{otherwise}
\end{cases}
\end{equation}

\subsubsection{Spectral Entropy VAD}

Spectral entropy measures spectral flatness:
\begin{equation}
H = -\sum_{k=0}^{K-1} P[k] \log_2(P[k])
\end{equation}

where $P[k] = |X[k]|^2 / \sum_j |X[j]|^2$ is the normalized power spectrum.

Normalized entropy:
\begin{equation}
H_{\text{norm}} = \frac{H}{\log_2(K)}
\end{equation}

Speech typically has $H_{\text{norm}} < 0.7$ (more structured than noise).

\subsubsection{Hangover Scheme}

Prevent speech clipping with hangover frames:
\begin{equation}
\text{VAD}_{\text{hangover}}[m] = \max_{k=0}^{H-1} \text{VAD}[m-k]
\end{equation}

where $H$ is the hangover length (typically 10 frames).

\subsection{Stage 3: Noise Estimation (MCRA)}

The Minima Controlled Recursive Averaging algorithm tracks the noise floor:

\subsubsection{Smoothed Power Spectrum}

\begin{equation}
S[m, k] = \alpha_s S[m-1, k] + (1-\alpha_s) |Y[m, k]|^2
\end{equation}

with $\alpha_s = 0.8$.

\subsubsection{Minimum Tracking}

\begin{equation}
S_{\min}[m, k] = \min(S_{\min}[m-1, k], S[m, k])
\end{equation}

Reset periodically (every $L=96$ frames):
\begin{equation}
S_{\min}[m, k] = S[m, k] \quad \text{if } m \mod L = 0
\end{equation}

\subsubsection{Speech Presence Probability}

\begin{equation}
I[m, k] = \begin{cases}
1 & \text{if } S[m, k] / S_{\min}[m, k] > \delta \\
0 & \text{otherwise}
\end{cases}
\end{equation}

with threshold $\delta = 5$.

\subsubsection{Noise Estimate Update}

\begin{equation}
\lambda_n[m, k] = \tilde{\alpha}[k] \lambda_n[m-1, k] + (1-\tilde{\alpha}[k]) |Y[m, k]|^2
\end{equation}

where $\tilde{\alpha}[k] = \alpha_d + (1-\alpha_d) p[k]$ adapts based on speech probability.

\subsection{Stage 4: Spectral Enhancement}

\subsubsection{Spectral Subtraction}

Basic power spectral subtraction:
\begin{equation}
|\hat{S}[k]|^2 = \max\left(|Y[k]|^2 - \alpha |\hat{N}[k]|^2, \beta |\hat{N}[k]|^2\right)
\end{equation}

with oversubtraction factor $\alpha = 2.0$ and spectral floor $\beta = 0.02$.

\subsubsection{Wiener Filter}

Optimal linear filter minimizing MSE:
\begin{equation}
H_{\text{Wiener}}[k] = \frac{\xi[k]}{1 + \xi[k]}
\end{equation}

where $\xi[k] = |S[k]|^2 / \lambda_n[k]$ is the a priori SNR.

\subsubsection{MMSE-STSA Estimator}

The Minimum Mean Square Error Short-Time Spectral Amplitude estimator:
\begin{equation}
\hat{A}[k] = G(\xi[k], \gamma[k]) \cdot |Y[k]|
\end{equation}

where the gain function involves modified Bessel functions:
\begin{equation}
G(\xi, \gamma) = \frac{\sqrt{\pi}}{2} \cdot \frac{\sqrt{\nu}}{\gamma} \cdot \exp\left(-\frac{\nu}{2}\right) \cdot \left[(1+\nu)I_0\left(\frac{\nu}{2}\right) + \nu I_1\left(\frac{\nu}{2}\right)\right]
\end{equation}

with $\nu = \xi\gamma/(1+\xi)$ and $\gamma[k] = |Y[k]|^2/\lambda_n[k]$ is the a posteriori SNR.

\textbf{Decision-Directed A Priori SNR:}
\begin{equation}
\xi[m, k] = \alpha_{\text{DD}} \frac{|\hat{S}[m-1, k]|^2}{\lambda_n[m-1, k]} + (1-\alpha_{\text{DD}}) \max(\gamma[m, k] - 1, 0)
\end{equation}

with $\alpha_{\text{DD}} = 0.98$.

\subsection{Stage 5: Voice Separation (Deep Attractor Network)}

\subsubsection{Network Architecture}

The Deep Attractor Network separates sources using embedding clustering:

\begin{table}[htbp]
\centering
\caption{Deep Attractor Network Architecture}
\label{tab:dan-architecture}
\begin{tabular}{lll}
\toprule
\textbf{Layer} & \textbf{Configuration} & \textbf{Output Shape} \\
\midrule
Input & Magnitude spectrogram & $(T, 257)$ \\
BLSTM Layer 1 & 300 hidden, bidirectional & $(T, 600)$ \\
BLSTM Layer 2 & 300 hidden, bidirectional & $(T, 600)$ \\
BLSTM Layer 3 & 300 hidden, bidirectional & $(T, 600)$ \\
BLSTM Layer 4 & 300 hidden, bidirectional & $(T, 600)$ \\
Embedding Layer & Linear projection & $(T, 257 \times 40)$ \\
Reshape & -- & $(T, 257, 40)$ \\
\bottomrule
\end{tabular}
\end{table}

\subsubsection{Attractor Computation}

For training with ground truth mask $M$:
\begin{equation}
\mathbf{A} = \frac{\sum_{t,f} M[t,f] \cdot \mathbf{V}[t,f]}{\sum_{t,f} M[t,f]}
\end{equation}

For inference using k-means:
\begin{equation}
\mathbf{A} = \text{KMeans}(\mathbf{V}.\text{flatten}(), k=2).\text{centroids}[0]
\end{equation}

\subsubsection{Mask Estimation}

Soft mask from embedding-attractor similarity:
\begin{equation}
\hat{M}[t,f] = \sigma(\langle \mathbf{V}[t,f], \mathbf{A} \rangle)
\end{equation}

\subsubsection{Training Loss}

\begin{equation}
\mathcal{L} = \frac{1}{TF} \sum_{t,f} \|M[t,f] - \hat{M}[t,f]\|^2
\end{equation}

\subsection{Stage 6: Dereverberation}

\subsubsection{Weighted Prediction Error (WPE)}

WPE models late reverberation as a linear prediction of past frames:
\begin{equation}
S[t,f] = Y[t,f] - \sum_{d=D}^{D+L-1} G_d[f] Y[t-d,f]
\end{equation}

where $D$ is the prediction delay (typically 3 frames) and $L$ is the filter length.

\subsection{Complete Pipeline Integration}

\begin{algorithm}
\caption{Voice Isolation Pipeline}
\begin{algorithmic}[1]
\REQUIRE Raw audio $x$
\ENSURE Isolated voice $\hat{s}$
\STATE $x \gets$ RemoveDC($x$)
\STATE $x \gets$ PreEmphasis($x$, $\alpha=0.97$)
\STATE $X \gets$ STFT($x$, window=512, hop=128)
\FOR{each frame $m$}
    \STATE vad[$m$] $\gets$ VAD($X[m]$)
    \STATE $\lambda_n[m] \gets$ MCRA\_Update($|X[m]|^2$, vad[$m$])
    \STATE $\hat{S}[m] \gets$ MMSE\_STSA($X[m]$, $\lambda_n[m]$)
\ENDFOR
\STATE $\hat{S} \gets$ DeepAttractorNetwork($\hat{S}$)
\STATE $\hat{S} \gets$ WPE\_Dereverb($\hat{S}$)
\STATE $\hat{s} \gets$ ISTFT($\hat{S}$)
\STATE $\hat{s} \gets$ DeEmphasis($\hat{s}$, $\alpha=0.97$)
\RETURN $\hat{s}$
\end{algorithmic}
\end{algorithm}

\section{Blender MCP Integration}

\subsection{Model Context Protocol}

The Model Context Protocol (MCP) provides a standardized interface for AI-tool integration \cite{anthropic2024mcp}.

\begin{figure}[htbp]
\centering
\fbox{\parbox{0.9\textwidth}{\centering
\vspace{2cm}
\textbf{[FIGURE PLACEHOLDER]}\\
MCP Architecture\\
Client-Server communication over JSON-RPC\\
\vspace{2cm}
}}
\caption{Model Context Protocol architecture for Blender integration.}
\label{fig:mcp-architecture}
\end{figure}

\subsubsection{Communication Protocol}

MCP uses JSON-RPC 2.0 over TCP (port 9876):

\begin{lstlisting}[language=json, caption=MCP Request Format]
{
    "jsonrpc": "2.0",
    "id": 1,
    "method": "tools/call",
    "params": {
        "name": "blender_create_mesh",
        "arguments": {
            "type": "cube",
            "name": "MyCube",
            "size": 2.0
        }
    }
}
\end{lstlisting}

\subsection{Tool Suite}

The MCP server exposes 24 operations:

\begin{table}[htbp]
\centering
\caption{Blender MCP Tool Categories}
\label{tab:mcp-tools}
\begin{tabular}{lp{8cm}}
\toprule
\textbf{Category} & \textbf{Operations} \\
\midrule
Object Creation & create\_mesh, create\_curve, create\_empty, create\_camera, create\_light \\
Transformation & translate, rotate, scale, apply\_transforms \\
Modifiers & add\_modifier, apply\_modifier, remove\_modifier \\
Materials & create\_material, assign\_material, set\_material\_property \\
Animation & insert\_keyframe, create\_action, set\_frame\_range \\
Export & export\_fbx, export\_gltf, export\_obj \\
Asset Import & import\_sketchfab, import\_polyhaven, generate\_hyper3d \\
\bottomrule
\end{tabular}
\end{table}

\subsection{Asset Source Integration}

\subsubsection{Multi-Source Fallback Chain}

\begin{enumerate}
    \item \textbf{Sketchfab}: 5M+ 3D models with licensing metadata
    \item \textbf{Poly Haven}: CC0 assets (materials, HDRIs, models)
    \item \textbf{Hyper3D Rodin}: AI-generated models from text/image
    \item \textbf{Hunyuan3D-2.0}: Multi-view diffusion 3D generation
\end{enumerate}

\subsubsection{Export Pipeline}

FBX export for game engine compatibility:

\begin{lstlisting}[language=Python, caption=FBX Export Configuration]
bpy.ops.export_scene.fbx(
    filepath=output_path,
    use_selection=True,
    apply_modifiers=True,
    mesh_smooth_type='FACE',
    use_armature_deform_only=True,
    add_leaf_bones=False,
    bake_anim=True,
    bake_anim_use_all_actions=False
)
\end{lstlisting}

% =============================================================================
% CHAPTER 4: EXPERIMENTAL RESULTS
% =============================================================================
\chapter{Experimental Results}

\section{Experimental Setup}

\subsection{Hardware Configuration}

\begin{table}[htbp]
\centering
\caption{Hardware Configuration}
\label{tab:hardware}
\begin{tabular}{ll}
\toprule
\textbf{Component} & \textbf{Specification} \\
\midrule
GPU & NVIDIA RTX 4090 (24GB VRAM) \\
CPU & AMD Ryzen 9 7950X \\
RAM & 64GB DDR5 \\
Storage & 2TB NVMe SSD \\
Training Time & 72 hours (VoxFormer full training) \\
\bottomrule
\end{tabular}
\end{table}

\subsection{Datasets}

\textbf{VoxFormer Training:}
\begin{itemize}
    \item LibriSpeech: 960 hours (train-clean-100, train-clean-360, train-other-500)
    \item Evaluation: dev-clean, dev-other, test-clean, test-other
\end{itemize}

\textbf{RAG Evaluation:}
\begin{itemize}
    \item 15,000 Blender documentation chunks
    \item 500 hand-crafted Q\&A pairs
    \item Ground truth annotations for RAGAS
\end{itemize}

\textbf{DSP Evaluation:}
\begin{itemize}
    \item VCTK: Multi-speaker clean speech
    \item DNS Challenge: Noisy speech with ground truth
    \item LibriMix: Multi-speaker mixtures
\end{itemize}

\section{VoxFormer Results}

\subsection{Word Error Rate}

\begin{table}[htbp]
\centering
\caption{VoxFormer Word Error Rate (\%)}
\label{tab:wer-results}
\begin{tabular}{lcccc}
\toprule
\textbf{Model} & \textbf{dev-clean} & \textbf{dev-other} & \textbf{test-clean} & \textbf{test-other} \\
\midrule
Whisper-small & 3.4 & 8.7 & 3.4 & 8.9 \\
Whisper-medium & 2.9 & 6.8 & 3.0 & 7.0 \\
Conformer-CTC & 3.1 & 7.2 & 3.2 & 7.5 \\
VoxFormer (ours) & \textbf{2.6} & \textbf{6.1} & \textbf{2.8} & \textbf{6.4} \\
\bottomrule
\end{tabular}
\end{table}

\begin{figure}[htbp]
\centering
\fbox{\parbox{0.9\textwidth}{\centering
\vspace{3cm}
\textbf{[FIGURE PLACEHOLDER]}\\
WER Comparison Bar Chart\\
VoxFormer vs baselines across test sets\\
\vspace{2cm}
}}
\caption{Word Error Rate comparison across LibriSpeech test sets.}
\label{fig:wer-comparison}
\end{figure}

\subsection{Training Convergence}

\begin{figure}[htbp]
\centering
\fbox{\parbox{0.9\textwidth}{\centering
\vspace{3cm}
\textbf{[FIGURE PLACEHOLDER]}\\
Training Loss and WER Curves\\
3-stage curriculum learning progression\\
\vspace{2cm}
}}
\caption{VoxFormer training convergence over 70 epochs with 3-stage curriculum.}
\label{fig:training-convergence}
\end{figure}

\subsection{Ablation Studies}

\begin{table}[htbp]
\centering
\caption{VoxFormer Ablation Study (test-clean WER \%)}
\label{tab:voxformer-ablation}
\begin{tabular}{lc}
\toprule
\textbf{Configuration} & \textbf{WER (\%)} \\
\midrule
Full VoxFormer & \textbf{2.8} \\
w/o RoPE (sinusoidal PE) & 3.2 \\
w/o SwiGLU (ReLU FFN) & 3.1 \\
w/o WavLM (mel features) & 4.1 \\
w/o Hybrid loss (CE only) & 3.4 \\
w/o Curriculum (direct 960h) & 3.5 \\
\bottomrule
\end{tabular}
\end{table}

\section{RAG System Results}

\subsection{Retrieval Metrics}

\begin{table}[htbp]
\centering
\caption{RAG Retrieval Performance}
\label{tab:rag-retrieval}
\begin{tabular}{lccc}
\toprule
\textbf{Method} & \textbf{MRR@10} & \textbf{Recall@10} & \textbf{Latency (ms)} \\
\midrule
Dense only (BGE-M3) & 0.72 & 0.81 & 45 \\
Sparse only (BM25) & 0.65 & 0.73 & 12 \\
Hybrid (RRF) & 0.79 & 0.87 & 62 \\
Hybrid + Reranking & \textbf{0.84} & \textbf{0.91} & 185 \\
\bottomrule
\end{tabular}
\end{table}

\subsection{RAGAS Evaluation}

\begin{table}[htbp]
\centering
\caption{RAGAS Evaluation Metrics}
\label{tab:ragas-results}
\begin{tabular}{lcccc}
\toprule
\textbf{Configuration} & \textbf{Faithfulness} & \textbf{Relevancy} & \textbf{Precision} & \textbf{Recall} \\
\midrule
Baseline RAG & 0.72 & 0.68 & 0.65 & 0.61 \\
Hybrid Retrieval & 0.78 & 0.74 & 0.72 & 0.69 \\
+ Cross-Encoder & 0.83 & 0.79 & 0.78 & 0.74 \\
+ Agentic Validation & \textbf{0.89} & \textbf{0.85} & \textbf{0.86} & \textbf{0.81} \\
\bottomrule
\end{tabular}
\end{table}

\subsection{Query Latency Analysis}

\begin{table}[htbp]
\centering
\caption{RAG Pipeline Latency Breakdown}
\label{tab:rag-latency}
\begin{tabular}{lcc}
\toprule
\textbf{Stage} & \textbf{Latency (ms)} & \textbf{Target (ms)} \\
\midrule
Query Embedding & 95 & 100 \\
Dense Search (HNSW) & 28 & 30 \\
Sparse Search (BM25) & 35 & 40 \\
RRF Fusion & 8 & 10 \\
Cross-Encoder (50 docs) & 580 & 600 \\
Context Assembly & 22 & 30 \\
LLM Generation & 1,850 & 2,000 \\
\midrule
\textbf{Total} & \textbf{2,618} & \textbf{2,810} \\
\bottomrule
\end{tabular}
\end{table}

\section{TTS and Lip-Sync Results}

\subsection{Latency Measurements}

\begin{table}[htbp]
\centering
\caption{TTS and Lip-Sync Latency}
\label{tab:tts-latency}
\begin{tabular}{lcc}
\toprule
\textbf{Component} & \textbf{Latency (ms)} & \textbf{Target (ms)} \\
\midrule
ElevenLabs TTFB & 72 & 75 \\
SadTalker Processing & 850 & 1,000 \\
MuseTalk Frame Generation & 28 & 33 \\
Total E2E (streaming) & 95 & 100 \\
\bottomrule
\end{tabular}
\end{table}

\subsection{Quality Metrics}

\begin{table}[htbp]
\centering
\caption{TTS Quality Metrics}
\label{tab:tts-quality}
\begin{tabular}{lcc}
\toprule
\textbf{Metric} & \textbf{ElevenLabs} & \textbf{Target} \\
\midrule
MOS (Mean Opinion Score) & 4.14 & 4.0 \\
Lip-Sync Error (LSE-D) & 7.2 & 8.0 \\
Lip-Sync Error (LSE-C) & 2.8 & 3.0 \\
FID (Face Quality) & 12.4 & 15.0 \\
\bottomrule
\end{tabular}
\end{table}

\section{DSP Voice Isolation Results}

\subsection{Signal-to-Noise Ratio Improvement}

\begin{table}[htbp]
\centering
\caption{SNR Improvement by Pipeline Stage}
\label{tab:snr-improvement}
\begin{tabular}{lcc}
\toprule
\textbf{Stage} & \textbf{SNR Improvement (dB)} & \textbf{Cumulative (dB)} \\
\midrule
Input (noisy) & 0 & 5.0 \\
After Spectral Subtraction & +4.2 & 9.2 \\
After MMSE-STSA & +3.8 & 13.0 \\
After DAN Separation & +5.1 & 18.1 \\
After Dereverberation & +1.9 & 20.0 \\
\bottomrule
\end{tabular}
\end{table}

\subsection{Separation Quality}

\begin{table}[htbp]
\centering
\caption{Source Separation Metrics on LibriMix}
\label{tab:separation-metrics}
\begin{tabular}{lccc}
\toprule
\textbf{Method} & \textbf{SI-SDRi (dB)} & \textbf{SDRi (dB)} & \textbf{PESQ} \\
\midrule
Spectral Subtraction & 8.2 & 8.5 & 2.1 \\
Wiener Filter & 9.4 & 9.8 & 2.3 \\
MMSE-STSA & 10.1 & 10.5 & 2.5 \\
Deep Attractor Network & \textbf{14.3} & \textbf{14.8} & \textbf{3.2} \\
\bottomrule
\end{tabular}
\end{table}

\begin{figure}[htbp]
\centering
\fbox{\parbox{0.9\textwidth}{\centering
\vspace{3cm}
\textbf{[FIGURE PLACEHOLDER]}\\
Spectrogram Comparison\\
Input mixture, separated speech, ground truth\\
\vspace{2cm}
}}
\caption{Spectrogram visualization of voice isolation pipeline stages.}
\label{fig:spectrogram-comparison}
\end{figure}

\section{Blender MCP Results}

\subsection{Task Completion Rate}

\begin{table}[htbp]
\centering
\caption{MCP Task Success Rate by Category}
\label{tab:mcp-success}
\begin{tabular}{lcc}
\toprule
\textbf{Category} & \textbf{Success Rate (\%)} & \textbf{Avg. Time (s)} \\
\midrule
Object Creation & 98.5 & 0.8 \\
Transformations & 99.2 & 0.3 \\
Modifiers & 96.8 & 1.2 \\
Materials & 94.5 & 2.1 \\
Animation & 93.2 & 3.5 \\
Export & 97.8 & 4.2 \\
Asset Import & 89.3 & 12.5 \\
\midrule
\textbf{Overall} & \textbf{95.6} & \textbf{3.5} \\
\bottomrule
\end{tabular}
\end{table}

\subsection{Productivity Impact}

\begin{table}[htbp]
\centering
\caption{Workflow Time Comparison}
\label{tab:productivity}
\begin{tabular}{lccc}
\toprule
\textbf{Task} & \textbf{Manual (min)} & \textbf{AI-Assisted (min)} & \textbf{Reduction (\%)} \\
\midrule
Create basic character & 45 & 12 & 73 \\
Apply materials/textures & 30 & 8 & 73 \\
Rig for animation & 60 & 18 & 70 \\
Export to Unity & 15 & 4 & 73 \\
\midrule
\textbf{Average} & \textbf{37.5} & \textbf{10.5} & \textbf{72} \\
\bottomrule
\end{tabular}
\end{table}

% =============================================================================
% CHAPTER 5: DISCUSSION
% =============================================================================
\chapter{Discussion}

\section{Analysis of Results}

\subsection{VoxFormer Performance}

The VoxFormer architecture achieves state-of-the-art results on LibriSpeech benchmarks, demonstrating the effectiveness of combining pre-trained acoustic representations with efficient Conformer-based temporal modeling. Key findings include:

\begin{enumerate}
    \item \textbf{WavLM Contribution}: The pre-trained WavLM encoder provides a 1.3\% absolute WER improvement compared to training from mel features, confirming the value of self-supervised pre-training for downstream ASR tasks.

    \item \textbf{RoPE Advantage}: Rotary Position Embeddings provide 0.4\% WER improvement over sinusoidal positional encoding while enabling better generalization to longer sequences.

    \item \textbf{Hybrid Loss Benefits}: The combination of CTC and cross-entropy losses improves convergence stability and reduces WER by 0.6\% compared to cross-entropy alone.

    \item \textbf{Curriculum Learning}: The 3-stage training curriculum improves final WER by 0.7\% while reducing total training time by 25\%.
\end{enumerate}

\subsection{RAG System Analysis}

The agentic RAG system demonstrates significant improvements over baseline retrieval approaches:

\begin{enumerate}
    \item \textbf{Hybrid Retrieval}: Combining dense and sparse retrieval with RRF fusion improves MRR@10 by 0.12 over dense-only retrieval, capturing both semantic similarity and exact keyword matches.

    \item \textbf{Cross-Encoder Reranking}: Adding cross-encoder reranking improves context precision by 0.14, crucial for reducing hallucination in generated responses.

    \item \textbf{Agentic Validation}: The self-correcting validation loop reduces hallucination rate from 12\% to under 5\%, with an average of 1.3 retrieval iterations per query.
\end{enumerate}

\subsection{DSP Pipeline Effectiveness}

The 6-stage DSP pipeline achieves robust voice isolation:

\begin{enumerate}
    \item \textbf{MCRA Noise Estimation}: Accurately tracks noise floor even during speech activity, enabling 4.2 dB SNR improvement through spectral subtraction alone.

    \item \textbf{MMSE-STSA Enhancement}: The decision-directed a priori SNR estimation reduces musical noise artifacts common in spectral subtraction.

    \item \textbf{Deep Attractor Network}: Provides an additional 5.1 dB SI-SDR improvement for multi-speaker scenarios, demonstrating the value of neural separation.
\end{enumerate}

\section{Limitations}

\subsection{VoxFormer Limitations}

\begin{itemize}
    \item \textbf{Computational Cost}: The 142M parameter model requires significant GPU memory (8GB minimum) for inference.
    \item \textbf{Technical Vocabulary}: Performance degrades on highly specialized technical terms not in training data.
    \item \textbf{Real-time Latency}: End-to-end inference latency of 180ms may be too high for some interactive applications.
\end{itemize}

\subsection{RAG System Limitations}

\begin{itemize}
    \item \textbf{Knowledge Freshness}: Static knowledge base requires manual updates for new Blender versions.
    \item \textbf{Complex Multi-hop Queries}: Performance degrades for queries requiring more than 3 reasoning hops.
    \item \textbf{Cross-Encoder Latency}: Reranking 50 candidates adds 580ms latency.
\end{itemize}

\subsection{DSP Pipeline Limitations}

\begin{itemize}
    \item \textbf{Single Microphone}: Current implementation is single-channel; beamforming would improve separation.
    \item \textbf{DAN Training Data}: Separation quality depends on similarity to training speakers.
    \item \textbf{Reverberant Environments}: WPE dereverberation struggles with RT60 > 800ms.
\end{itemize}

\section{Future Work}

\subsection{Short-term Improvements}

\begin{enumerate}
    \item \textbf{Model Distillation}: Create smaller VoxFormer variants for edge deployment
    \item \textbf{Streaming Inference}: Implement chunk-based processing for real-time STT
    \item \textbf{RAG Knowledge Updates}: Automated documentation ingestion pipeline
    \item \textbf{Multi-channel DSP}: Add beamforming for multi-microphone arrays
\end{enumerate}

\subsection{Long-term Research Directions}

\begin{enumerate}
    \item \textbf{End-to-End Integration}: Unified model for speech understanding, retrieval, and response generation
    \item \textbf{Personalization}: User-adaptive models that learn individual speech patterns and preferences
    \item \textbf{Multi-modal Understanding}: Integration of visual scene understanding for context-aware assistance
    \item \textbf{Procedural Content Generation}: AI-driven 3D asset creation beyond retrieval and modification
\end{enumerate}

% =============================================================================
% CHAPTER 6: CONCLUSION
% =============================================================================
\chapter{Conclusion}

This report presented the 3D Game Generation AI Assistant, a comprehensive system integrating five synergistic components for voice-controlled 3D game development. The key contributions and achievements are:

\section{Summary of Contributions}

\begin{enumerate}
    \item \textbf{VoxFormer Architecture}: A novel encoder-decoder model achieving 2.8\% WER on LibriSpeech test-clean, surpassing comparable-size baselines through innovative combination of WavLM pre-training, Zipformer temporal modeling, and hybrid CTC-attention training.

    \item \textbf{Agentic RAG System}: A 7-layer retrieval-augmented generation framework with hybrid dense-sparse retrieval, cross-encoder reranking, and self-correcting validation achieving 0.89 faithfulness and 0.86 context precision on domain-specific queries.

    \item \textbf{Real-time Avatar Pipeline}: Integration of ElevenLabs TTS with SadTalker and MuseTalk for sub-100ms lip-synchronized speech generation with photorealistic quality.

    \item \textbf{Low-level DSP Implementation}: From-scratch signal processing pipeline achieving 15dB+ SNR improvement through MCRA noise estimation, MMSE-STSA enhancement, and Deep Attractor Networks.

    \item \textbf{Blender MCP Integration}: Comprehensive 24-operation tool suite enabling 73\% reduction in manual 3D asset creation time.
\end{enumerate}

\section{Impact}

The integrated system demonstrates the feasibility of voice-controlled 3D development workflows, with significant productivity improvements measured across various tasks. The modular architecture enables independent component upgrades while maintaining system coherence.

\section{Final Remarks}

This work establishes a foundation for AI-assisted creative tools that understand natural language, access relevant knowledge, and execute complex operations in professional software. As AI capabilities continue to advance, we anticipate such systems becoming indispensable aids for creative professionals across industries.

% =============================================================================
% BIBLIOGRAPHY
% =============================================================================
\printbibliography[heading=bibintoc, title={References}]

% =============================================================================
% APPENDICES
% =============================================================================
\appendix

\chapter{Mathematical Derivations}

\section{CTC Forward-Backward Algorithm}

The forward variable $\alpha_t(s)$ is computed recursively:

\textbf{Initialization:}
\begin{align}
\alpha_1(1) &= p_1(\epsilon) \\
\alpha_1(2) &= p_1(y_1) \\
\alpha_1(s) &= 0, \quad s > 2
\end{align}

\textbf{Recursion:}
\begin{equation}
\alpha_t(s) = \begin{cases}
(\alpha_{t-1}(s) + \alpha_{t-1}(s-1)) \cdot p_t(y'_s) & \text{if } y'_s = \epsilon \text{ or } y'_s = y'_{s-2} \\
(\alpha_{t-1}(s) + \alpha_{t-1}(s-1) + \alpha_{t-1}(s-2)) \cdot p_t(y'_s) & \text{otherwise}
\end{cases}
\end{equation}

\textbf{Termination:}
\begin{equation}
P(\mathbf{y}|\mathbf{x}) = \alpha_T(|y'|) + \alpha_T(|y'|-1)
\end{equation}

The backward variable $\beta_t(s)$ is computed similarly from $t=T$ to $t=1$.

\section{MMSE-STSA Derivation}

Starting from the Bayesian estimator:
\begin{equation}
\hat{A} = \mathbb{E}[A | Y] = \int_0^\infty A \cdot p(A|Y) dA
\end{equation}

Under Gaussian assumptions for speech and noise:
\begin{equation}
p(A|Y) = \frac{p(Y|A) p(A)}{p(Y)}
\end{equation}

After integration (see Ephraim \& Malah, 1984):
\begin{equation}
G(\xi, \gamma) = \frac{\Gamma(1.5)\sqrt{\nu}}{\gamma} \exp\left(-\frac{\nu}{2}\right) \left[(1+\nu)I_0\left(\frac{\nu}{2}\right) + \nu I_1\left(\frac{\nu}{2}\right)\right]
\end{equation}

where $\Gamma(1.5) = \sqrt{\pi}/2$.

\section{RoPE Derivation}

For 2D rotation, the rotation matrix is:
\begin{equation}
R_\theta = \begin{pmatrix} \cos\theta & -\sin\theta \\ \sin\theta & \cos\theta \end{pmatrix}
\end{equation}

Extending to $d$ dimensions with different frequencies:
\begin{equation}
R_{\Theta,m}^d = \text{diag}(R_{m\theta_1}, R_{m\theta_2}, \ldots, R_{m\theta_{d/2}})
\end{equation}

The key property ensuring relative position encoding:
\begin{equation}
\langle R_{\Theta,m}^d \mathbf{q}, R_{\Theta,n}^d \mathbf{k} \rangle = \mathbf{q}^T (R_{\Theta,m}^d)^T R_{\Theta,n}^d \mathbf{k} = \mathbf{q}^T R_{\Theta,n-m}^d \mathbf{k}
\end{equation}

\chapter{Implementation Details}

\section{VoxFormer Hyperparameters}

\begin{table}[htbp]
\centering
\caption{Complete VoxFormer Hyperparameters}
\label{tab:voxformer-hyperparams}
\begin{tabular}{ll}
\toprule
\textbf{Parameter} & \textbf{Value} \\
\midrule
\multicolumn{2}{l}{\textit{Audio Frontend}} \\
Sample rate & 16,000 Hz \\
Window size & 400 samples (25ms) \\
Hop size & 160 samples (10ms) \\
FFT size & 512 \\
Mel filterbank channels & 80 \\
\midrule
\multicolumn{2}{l}{\textit{WavLM Encoder}} \\
Hidden dimension & 768 \\
Attention heads & 12 \\
Layers & 12 \\
\midrule
\multicolumn{2}{l}{\textit{Zipformer Encoder}} \\
Blocks & 6 \\
Hidden dimension & 512 \\
Attention heads & 8 \\
Convolution kernel size & 31 \\
\midrule
\multicolumn{2}{l}{\textit{Decoder}} \\
Layers & 6 \\
Hidden dimension & 512 \\
Attention heads & 8 \\
Vocabulary size & 5,000 \\
\midrule
\multicolumn{2}{l}{\textit{Training}} \\
Optimizer & AdamW \\
Peak learning rate & 1e-3 \\
Weight decay & 0.01 \\
Warmup steps & 10,000 \\
CTC weight ($\lambda$) & 0.3 \\
Label smoothing & 0.1 \\
SpecAugment freq masks & 2 (width 27) \\
SpecAugment time masks & 2 (width 100) \\
\bottomrule
\end{tabular}
\end{table}

\section{RAG System Configuration}

\begin{table}[htbp]
\centering
\caption{RAG System Configuration Parameters}
\label{tab:rag-config}
\begin{tabular}{ll}
\toprule
\textbf{Parameter} & \textbf{Value} \\
\midrule
\multicolumn{2}{l}{\textit{Embedding}} \\
Model & BAAI/bge-m3 \\
Dimension & 4,096 \\
Batch size & 32 \\
\midrule
\multicolumn{2}{l}{\textit{Vector Index (HNSW)}} \\
$m$ (connections) & 16 \\
ef\_construction & 200 \\
ef\_search & 100 \\
\midrule
\multicolumn{2}{l}{\textit{BM25}} \\
$k_1$ & 1.5 \\
$b$ & 0.75 \\
\midrule
\multicolumn{2}{l}{\textit{RRF}} \\
$k$ constant & 60 \\
\midrule
\multicolumn{2}{l}{\textit{Reranker}} \\
Model & cross-encoder/ms-marco-MiniLM-L-6-v2 \\
Candidates & 50 \\
Output & Top 10 \\
\midrule
\multicolumn{2}{l}{\textit{Validation}} \\
Max retries & 3 \\
Grounding threshold & 0.7 \\
\bottomrule
\end{tabular}
\end{table}

\chapter{Additional Figures}

\begin{figure}[htbp]
\centering
\fbox{\parbox{0.9\textwidth}{\centering
\vspace{4cm}
\textbf{[FIGURE PLACEHOLDER]}\\
Complete System Integration Diagram\\
All 5 components with data flow\\
\vspace{3cm}
}}
\caption{Complete 3D Game AI Assistant system architecture showing all component interconnections.}
\label{fig:complete-system}
\end{figure}

\begin{figure}[htbp]
\centering
\fbox{\parbox{0.9\textwidth}{\centering
\vspace{4cm}
\textbf{[FIGURE PLACEHOLDER]}\\
User Interface Screenshots\\
Voice command input, RAG response, Avatar output\\
\vspace{3cm}
}}
\caption{User interface demonstration showing end-to-end voice-controlled 3D asset creation workflow.}
\label{fig:ui-screenshots}
\end{figure}

\end{document}
